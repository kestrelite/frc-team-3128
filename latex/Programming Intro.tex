\documentclass[a4paper]{article}
\usepackage[english]{babel}
\usepackage{amsmath}
\usepackage{graphicx}
\usepackage{color}
\usepackage{listings}
\usepackage[margin=1in]{geometry}
\usepackage[T1]{fontenc}
\title{FRC Programming Introduction v2.0}
\author{Noah Sutton-Smolin\\Yousuf Soliman\\Kian Sheik}

\definecolor{dkgreen}{rgb}{0,0.6,0}
\definecolor{gray}{rgb}{0.5,0.5,0.5}
\definecolor{mauve}{rgb}{0.58,0,0.82}
\definecolor{black}{rgb}{0,0,0}

\lstset{frame=tb,
  language=Java,
  aboveskip=3mm,
  belowskip=3mm,
  showstringspaces=false,
%  columns=flexible,
  basicstyle={\small\ttfamily},
  numbers=none,
  numberstyle=\tiny\color{gray},
  keywordstyle=\color{blue},
  commentstyle=\color{black},
  stringstyle=\color{mauve},
  breaklines=true,
  breakatwhitespace=true
  tabsize=2
}

\begin{document}
\maketitle

\section{Introduction}

This introduction to FRC programming assumes pre-existing Java programming knowledge and basic syntax understanding.

If you do not have a fundamental understanding of Java, that is fine, however, to write code for FRC, you will need to learn the basic syntax. You may want to take a look at these links, as well as look into Java on your own:

\begin{lstlisting}
http://docs.oracle.com/javase/tutorial/java/
http://www.tutorialspoint.com/java/
http://chortle.ccsu.edu/CS151/cs151java.html
\end{lstlisting}

Programming for FRC will require a configured Netbeans and Subversion environment. This document is designed to walk you through the setup process. While a Windows computer is not \textit{required} for FRC development, it is highly recommended. Once you have completed the FRC introduction sets, you may be given commit access to the FRC repository.

\section{Setting up the NetBeans environment}

\textbf{Once you have completed these steps, it is highly recommended that you download the \lstinline{svn/doc} folder and look through it.}

The entire API we have written has javadoc, which means that hovering over a function will show the description, parameters, and return values.

\subsection{Installing and configuring the FRC environment}
\begin{enumerate}
\item{Download and install Netbeans 7.3.1 \textit{specifically} (you will need the latest Java Development Kit (JDK))}
\item{Go to $Tools \rightarrow Plugins \rightarrow Settings \rightarrow Add$}
\item{Enter ``FRC Java'' under New Provider}
\item{Enter the following link under URL: \lstinline{http://first.wpi.edu/FRC/java/netbeans/update/Release/updates.xml}}
\item{Go to the Available Plugins tab and click Check for Updates}
\item{Find all plugins that start with ``FRC'' and select them. Install those.}
\item{Close the Plugins dialog}
\item{Go to $Tools\rightarrow Options\rightarrow Miscellaneous\rightarrow FRC\ Configuration$ and enter 3128 (or your team number) under ``Team Number''}
\item{Click OK}
\end{enumerate}

Once you have checked out the introductory tasks, you can open them by going to $File\rightarrow Open\ Project\ldots$ and finding the Java project.

\subsection{Checking out projects with Subversion in NetBeans}
\begin{enumerate}
\item{Go to $Team\rightarrow Subversion\rightarrow Checkout$ (there may be an option to check out without going through the Subversion menu).}
\item{Enter the repository URL:\lstinline{https://frc-team-3128.googlecode.com/svn/} and leave the username and password blank.}
\item{Select and open all projects.}
\end{enumerate}

The NetBeans Subversion menu is available by right clicking in the Projects pane; it is available at every scope. \textbf{Please do not commit private.xml.}

\section{Running the introductory tasks}

The tutorials are part of the intro projects; if you checked them out to the standard directory, yours will be located in $C:\textbackslash Users\textbackslash\{username\}\textbackslash My Documents\textbackslash NetBeansProjects\textbackslash intro\textbackslash$ (Windows 7). From there, each of the lesson sets are in their own folder. Each lesson is in its own text file.

All of the information about the introductory tasks is located in \lstinline{latex/Programming Tasks.pdf}. There are five subdirectories in the intro folder. Each subdirectory contains its own refined project, each corresponding to the section headings in \lstinline{Programming Tasks.pdf}. 

Once you've worked through one of these, call one of the programmers over and we will look through and comment on your code. Once you've completed a task, you are free to move on to the next one. 

Welcome to FRC Robotics Programming! \textbf{Remember: Ask questions, get answers!} If you aren't asking questions, you're doing something wrong.

\end{document}

